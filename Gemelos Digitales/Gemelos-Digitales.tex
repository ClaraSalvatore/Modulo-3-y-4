\documentclass{article}

\usepackage{arxiv}

\usepackage[utf8]{inputenc} % allow utf-8 input
\usepackage[T1]{fontenc}    % use 8-bit T1 fonts
\usepackage{lmodern}        % https://github.com/rstudio/rticles/issues/343
\usepackage{hyperref}       % hyperlinks
\usepackage{url}            % simple URL typesetting
\usepackage{booktabs}       % professional-quality tables
\usepackage{amsfonts}       % blackboard math symbols
\usepackage{nicefrac}       % compact symbols for 1/2, etc.
\usepackage{microtype}      % microtypography
\usepackage{graphicx}

\title{Gemelos Digitales en viñedoS}

\author{
    MDEVOS
   \\
    Ingeniería Industrial \\
    Universidad Nacional De Cuyo (Delye, López, Salvatore, Spigolon,
Marini y Trad) \\
   \\
  \texttt{} \\
  }


% tightlist command for lists without linebreak
\providecommand{\tightlist}{%
  \setlength{\itemsep}{0pt}\setlength{\parskip}{0pt}}



\begin{document}
\maketitle


\begin{abstract}

\end{abstract}


\hypertarget{introducciuxf3n}{%
\section{Introducción}\label{introducciuxf3n}}

Un gemelo digital es una réplica virtual de un objeto, sistema o proceso
físico que se utiliza para simular, analizar y optimizar su
comportamiento en el mundo real. Este concepto se basa en la creación de
un modelo digital detallado y dinámico que refleja fielmente las
características y el funcionamiento del objeto o sistema real. Los
gemelos digitales pueden recopilar datos en tiempo real a través de
sensores y otras tecnologías IoT (Internet de las Cosas), lo que permite
una monitorización continua y una gestión proactiva.

\hypertarget{gemelos-digitales-en-viuxf1edos-para-programaciuxf3n-automuxe1tica-de-riego}{%
\section{Gemelos Digitales en viñedos para programación automática de
riego}\label{gemelos-digitales-en-viuxf1edos-para-programaciuxf3n-automuxe1tica-de-riego}}

Los gemelos digitales en viñedos son una aplicación avanzada de esta
tecnología que permite optimizar la gestión de los viñedos,
especialmente en lo que respecta a la programación automática de riegos.

\label{sec:headings}

\ref{sec:headings}.

\hypertarget{funcionamiento-de-los-gemelos-digitales-en-viuxf1edos}{%
\subsection{Funcionamiento de los Gemelos Digitales en
Viñedos}\label{funcionamiento-de-los-gemelos-digitales-en-viuxf1edos}}

Sensores: Se instalan sensores en el viñedo para recopilar datos en
tiempo real sobre diversos parámetros como la humedad del suelo,
temperatura, humedad del aire, nivel de nutrientes, y más.

Drones y Satélites: Se utilizan drones y satélites para obtener imágenes
y datos sobre el estado del viñedo, la salud de las plantas y otros
factores ambientales.

\hypertarget{modelado-del-viuxf1edo}{%
\subsubsection{Modelado del Viñedo:}\label{modelado-del-viuxf1edo}}

Modelo Digital: Se crea un modelo digital del viñedo que incluye todos
los datos recopilados. Este modelo es una réplica virtual que simula las
condiciones del viñedo.

Análisis de Datos: Los datos en tiempo real se integran en el modelo
digital, permitiendo un análisis continuo y preciso del estado del
viñedo.

\hypertarget{simulaciuxf3n-y-predicciuxf3n}{%
\paragraph{Simulación y
predicción}\label{simulaciuxf3n-y-predicciuxf3n}}

Simulaciones: El gemelo digital permite simular diferentes escenarios de
riego, teniendo en cuenta factores como el clima, la previsión
meteorológica y las necesidades específicas de las plantas.
Predicciones: Utilizando algoritmos de inteligencia artificial y
aprendizaje automático, el gemelo digital puede predecir cómo
responderán las plantas a diferentes regímenes de riego.

\hypertarget{programaciuxf3n-automuxe1tica-de-riegos}{%
\subparagraph{Programación Automática de
Riegos}\label{programaciuxf3n-automuxe1tica-de-riegos}}

Decisiones Basadas en Datos: La programación del riego se basa en datos
precisos y en tiempo real. El sistema puede decidir cuándo y cuánto
regar cada sección del viñedo. Automatización: Los sistemas de riego
automatizados pueden ser controlados directamente por el gemelo digital,
ajustando el riego de manera autónoma para optimizar el uso del agua y
mejorar la salud de las plantas.

\hypertarget{beneficios-de-los-gemelos-digitales-en-viuxf1edos}{%
\subsection{Beneficios de los Gemelos Digitales en
Viñedos}\label{beneficios-de-los-gemelos-digitales-en-viuxf1edos}}

\hypertarget{eficiencia-en-el-uso-del-agua}{%
\subsubsection{Eficiencia en el uso del
agua}\label{eficiencia-en-el-uso-del-agua}}

Riego Preciso: Se reduce el desperdicio de agua al regar solo cuando y
donde es necesario. Sostenibilidad: La optimización del riego contribuye
a prácticas agrícolas más sostenibles y responsables con el medio
ambiente.

\hypertarget{mejora-de-la-calidad-de-la-uva}{%
\subsubsection{Mejora de la Calidad de la
Uva}\label{mejora-de-la-calidad-de-la-uva}}

Condiciones Óptimas: Al asegurar que las plantas reciban la cantidad
adecuada de agua, se pueden mantener condiciones óptimas para el
crecimiento y la maduración de las uvas. Reducción de Estrés Hídrico:
Las plantas experimentan menos estrés hídrico, lo que puede mejorar la
calidad y el rendimiento de la cosecha.

\hypertarget{reducciuxf3n-de-costos}{%
\subsubsection{Reducción de Costos}\label{reducciuxf3n-de-costos}}

Ahorro de Recursos: Menos consumo de agua y energía para el riego.

Mantenimiento Predictivo: Identificación de problemas potenciales antes
de que se conviertan en fallos costosos.

\hypertarget{sostenibilidad-y-responsabilidad-ambiental}{%
\subsubsection{Sostenibilidad y Responsabilidad
Ambiental}\label{sostenibilidad-y-responsabilidad-ambiental}}

Reducción de Impacto Ambiental: El uso eficiente del agua y otros
recursos naturales contribuye a una agricultura más sostenible.
Adaptación al Cambio Climático: Mejora la resiliencia del viñedo frente
a las condiciones climáticas variables.

\hypertarget{conclusiuxf3n}{%
\subsection{Conclusión}\label{conclusiuxf3n}}

La implementación de gemelos digitales en viñedos representa una
revolución en la viticultura moderna. Al proporcionar una visión
detallada y en tiempo real del estado del viñedo, y al permitir la
simulación y optimización de diferentes estrategias de riego, los
gemelos digitales ayudan a los viticultores a gestionar sus viñedos de
manera más eficiente, sostenible y rentable. La optimización del uso del
agua, la mejora de la calidad de las uvas, la reducción de costos
operativos y la capacidad de tomar decisiones informadas basadas en
datos son solo algunos de los beneficios que esta tecnología puede
ofrecer. En un mundo donde los recursos naturales son cada vez más
limitados y el cambio climático es una realidad, los gemelos digitales
se presentan como una herramienta esencial para la agricultura del
futuro.

\bibliographystyle{unsrt}
\bibliography{references.bib}


\end{document}
